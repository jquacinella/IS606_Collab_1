\documentclass[]{article}
\usepackage[T1]{fontenc}
\usepackage{lmodern}
\usepackage{amssymb,amsmath}
\usepackage{ifxetex,ifluatex}
\usepackage{fixltx2e} % provides \textsubscript
% use upquote if available, for straight quotes in verbatim environments
\IfFileExists{upquote.sty}{\usepackage{upquote}}{}
\ifnum 0\ifxetex 1\fi\ifluatex 1\fi=0 % if pdftex
  \usepackage[utf8]{inputenc}
\else % if luatex or xelatex
  \ifxetex
    \usepackage{mathspec}
    \usepackage{xltxtra,xunicode}
  \else
    \usepackage{fontspec}
  \fi
  \defaultfontfeatures{Mapping=tex-text,Scale=MatchLowercase}
  \newcommand{\euro}{€}
\fi
% use microtype if available
\IfFileExists{microtype.sty}{\usepackage{microtype}}{}
\usepackage[margin=1in]{geometry}
\usepackage{graphicx}
% Redefine \includegraphics so that, unless explicit options are
% given, the image width will not exceed the width of the page.
% Images get their normal width if they fit onto the page, but
% are scaled down if they would overflow the margins.
\makeatletter
\def\ScaleIfNeeded{%
  \ifdim\Gin@nat@width>\linewidth
    \linewidth
  \else
    \Gin@nat@width
  \fi
}
\makeatother
\let\Oldincludegraphics\includegraphics
{%
 \catcode`\@=11\relax%
 \gdef\includegraphics{\@ifnextchar[{\Oldincludegraphics}{\Oldincludegraphics[width=\ScaleIfNeeded]}}%
}%
\ifxetex
  \usepackage[setpagesize=false, % page size defined by xetex
              unicode=false, % unicode breaks when used with xetex
              xetex]{hyperref}
\else
  \usepackage[unicode=true]{hyperref}
\fi
\hypersetup{breaklinks=true,
            bookmarks=true,
            pdfauthor={Brian C., James Q., Rohan F., Sharad G.},
            pdftitle={Sandwich Tycoon},
            colorlinks=true,
            citecolor=blue,
            urlcolor=blue,
            linkcolor=magenta,
            pdfborder={0 0 0}}
\urlstyle{same}  % don't use monospace font for urls
\setlength{\parindent}{0pt}
\setlength{\parskip}{6pt plus 2pt minus 1pt}
\setlength{\emergencystretch}{3em}  % prevent overfull lines
\setcounter{secnumdepth}{0}

\title{Sandwich Tycoon}
\author{Brian C., James Q., Rohan F., Sharad G.}
\date{}

\begin{document}

\begin{center}
\huge Sandwich Tycoon \\[0.2cm]
\end{center}
\begin{center}
\large \emph{Brian C., James Q., Rohan F., Sharad G.}\\[0.1cm]
\end{center}
\normalsize


\subsubsection{Preliminary Analysis}\label{preliminary-analysis}

After plotting histograms and scatterplots of the historical data, it
was evident that:

\begin{itemize}
\itemsep1pt\parskip0pt\parsep0pt
\item
  Demand can be modeled as a random variable.
\item
  Very often sandwich demand exceeded supply.
\item
  Daily sandwich demand was independent of previous demand.
\item
  Customer orders are independent of each other during the course of the
  day.
\item
  No obvious long-term demand trend upwards or downwards (linear
  regression, $a<.02$ and $r2<.0001$ for all types)
\end{itemize}

\subsubsection{Initial Graphs}\label{initial-graphs}

\includegraphics{./IS606_Sandwich_files/figure-latex/unnamed-chunk-11.pdf}
\includegraphics{./IS606_Sandwich_files/figure-latex/unnamed-chunk-12.pdf}
\includegraphics{./IS606_Sandwich_files/figure-latex/unnamed-chunk-13.pdf}

\subsubsection{Objective}\label{objective}

To maximize total sandwich profits over a 130-day period by estimating
probability of demand and producing a fixed or variable quantity of
supply.

\subsubsection{General Strategy}\label{general-strategy}

Sandwiches sold is a discrete variable and therefore can be estimated
using a probability mass function. We forecasted demand using two
different probability distributions. The first was to use the historical
frequency: This distribution would match the probability ($X=x$) of the
preceding period.

The second one is based on the frequency distribtutions suggest a
probability distribution. Since, there is no constraint on the number of
events and the outcomes are independent, we chose Poisson distribution
as a candidate. In order to user Poisson distribution, we calculated
lambda from historica data that gave us sandwich demand per day.

\includegraphics{./IS606_Sandwich_files/figure-latex/unnamed-chunk-2.pdf}

James previously supplied sandwiches at a mostly fixed amount. We used a
fixed supply model for the historical distribution but modeled the
poisson distribution under both fixed and variable supply assumptions.

A 130-day period was chosen to match the timeline of the given data.
This allows for direct benchmarking (given below assumptions) against 1)
profits that James actually made in the preceding 130 days, and 2) a
gold standard profit margin that was achievable over 130 days if supply
always met demand every day.

\subsubsection{General Assumptions}\label{general-assumptions}

\begin{itemize}
\itemsep1pt\parskip0pt\parsep0pt
\item
  Demand for each sandwich type is independent. What a customer orders
  is independent of what was ordered before.
\item
  Each customer only counts towards demand of one sandwich type.
  Therefore, if a customer wanted ham but it was sold out and turkey was
  bought instead, demand would count as 1 ham, 0 turkey. This means the
  sum of total demand equals the total number of customers who visited
  on a given day
\item
  Future demand will closely match historical demand. Again, there was
  no evident long-term trend and we have no information to assume a
  drastic drop or growth over the next 130 days (e.g.~more people in the
  building, other competition, vegan explosion, swine flu epidemic)
\item
  There are no added fixed costs to increasing supply (e.g.~hiring
  helpers, more preparation space/tools)
\item
  Supply goes to waste if not sold in a day. We vary this assumption in
  our second Poisson distribution model in that unsold sandwiches are
  reused (and thus increase future supply).
\end{itemize}

\subsubsection{Profit Results}\label{profit-results}

\paragraph{A) Previously achieved -
\$12,828}\label{a-previously-achieved---12828}

Given James' actual supply and demand over the 130-day period, he
achieved the following:

\begin{table}[!htbp]
  \label{} 
\begin{tabular}{@{\extracolsep{5pt}} ccccc} 
\\[-1.8ex]\hline 
\hline \\[-1.8ex] 
type & revenue & cost & profit \\ 
\hline \\[-1.8ex] 
ham & $12,012$ & $7,175$ & $4,837$ \\ 
turkey & $14,066$ & $8,960$ & $5,106$ \\ 
veggie & $5,960$ & $3,075$ & $2,885$ \\ 
total & $32,038$ & $19,210$ & $12,828$ \\ 
\hline \\[-1.8ex] 
\end{tabular} 
\end{table}

\paragraph{B) Historical Probability Distribution -
\$13,858}\label{b-historical-probability-distribution---13858}

We used historical frequency of each demand amount to determine the
probability ($X=x$) of each sandwich sold on a given day. With this
probability distribution, we simulated 10,000 trials over a 130-day
period to get our demand estimate. Under our assumption of fixed supply,
we calculated the revenue, cost, and profit for each fixed number of
sandwiches produced (over the demand range of each sandwich type).

The results demonstrate that the optimal fixed number of sandwiches to
supply per day is equal to the expected value, which under a specific
frequency distribution is the highest frequency value (ham: $n=15$
$p=0.123$, turkey: $n=20$ $p=0.1$, veggie: $n=13$ $p=.138$).

\begin{table}[!htbp]
  \label{} 
\begin{tabular}{@{\extracolsep{5pt}} ccccc} 
\\[-1.8ex]\hline 
\hline \\[-1.8ex] 
type & revenue & cost & profit \\ 
\hline \\[-1.8ex] 
ham & $11,765$ & $6,825$ & $4,940$ \\ 
turkey & $16,003$ & $10,400$ & $5,603$ \\ 
veggie & $7,540$ & $4,225$ & $3,315$ \\ 
total & $35,308$ & $21,450$ & $13,858$ \\ 
\hline \\[-1.8ex] 
\end{tabular} 
\end{table}

\textbf{(insert charts - frequency distribution)}

\paragraph{C) Poisson Distribution - Fixed
Supply}\label{c-poisson-distribution---fixed-supply}

\textbf{include that lambda = expected value = optimum supply level}

\subparagraph{Without Storage (unsold sandwiches are
wasted)}\label{without-storage-unsold-sandwiches-are-wasted}

\includegraphics{./IS606_Sandwich_files/figure-latex/unnamed-chunk-4.pdf}

\subparagraph{With Storage (unsold sandwiches are put back into
supply)}\label{with-storage-unsold-sandwiches-are-put-back-into-supply}

\includegraphics{./IS606_Sandwich_files/figure-latex/unnamed-chunk-5.pdf}

\paragraph{D) Poisson Distribution - Variable
Supply}\label{d-poisson-distribution---variable-supply}

\includegraphics{./IS606_Sandwich_files/figure-latex/unnamed-chunk-6.pdf}

\paragraph{E) Gold Standard - \$17,631}\label{e-gold-standard---17631}

This is the profit that would have been made if supply=demand each day
so that no sandwich was wasted and every customer was satisfied.
Comparing the previous methods as percent of gold standard achieved, we
see the poisson variable supply model yielded XX\% while the poisson
fixed model and historical probability model yielded YY\% and $78.6\%$,
respectively.

\subsubsection{Recommendations}\label{recommendations}

TODO: be more exact

\begin{itemize}
\itemsep1pt\parskip0pt\parsep0pt
\item
  If fixed supply, produce at rate of expected value
\item
  Poisson distribution fit historical data very well - can use as
  distribution function going forward
\item
  Consider investing in fridge, etc. to prolong product shelf life
\item
  Being able to carry over supply day-to-day greatly increases expected
  profit
\item
  If overestimating demand, we recommend veggie because the profit
  margin is the same as turkey but the cost is the least.
\item
  If underestimate demand, we recommend turkey because the cost is the
  highest but the profit is same as veggie.
\end{itemize}

\subsubsection{Limitations}\label{limitations}

\begin{itemize}
\itemsep1pt\parskip0pt\parsep0pt
\item
  Simple model assuming external factors not changing
\item
  Will historical demand \textasciitilde{} Future demand?
\item
  Covariance. If no turkey is produced, could some/all switch to higher
  margin ham?
\item
  Fixed supply costs in real world would likely increase
\item
  In variable model, 3-day old sandwich is as `desirable' as fresh one
\end{itemize}

\end{document}
